In dieser Arbeit werden Porphyrine und ihre Lumineszenz-Eigneschaften besprochen. Lumineszenz ist ein wichtige Eigenschaft in der Analytik und wird beispielsweise für optische Sensoren angewendet. Zur Lumineszenz gehören verschiedene Arten wie Photolumineszenz, Thermolumineszenz oder Chemi- oder Biolumineszenz. Hier werden auch die verschiedenen Übergänge zwischen elektronischen Zuständen (internal conversion, Fluoreszenz, Phosphoreszenz) beschrieben.
Porphyrine bilden eine essentiele Gruppe in  der Natur (Hämgruppe, Chlorophyll, Vitamin B12) und verschiedene wissenschaftlichen Bereichen und können als Indikatoren eingesetzt werden. Für die Synthese von längerwelligen Porphyrinen eignet sich die Scholl-Reaktion, die in letzter Zeit wieder an Bedeutung gewonnen hat.
\\Ziel dieser Arbeit ist es, verschiedene Synthesen von Benzoporphyrinen durchzuführen und die Eigenschaften dieser zu charakterisieren.
Die Synthesemethoden reichen von Kondensation, über Schmelzreaktionen bis zur Scholl- Reaktion. Zur Reinigung wurden meist Extraktionen oder Säulenchromatographie mit Kieselgel oder Aluminiumoxid verwendet. Zur Charakterisierung standen $H^1$-NMR-Spektroskopie, UV-Vis-Spektroskopie oder Fluoreszenzspektroskopie zur Verfügung.