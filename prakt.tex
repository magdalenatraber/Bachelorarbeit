\section{5-(pentafluorphenyl)bis(phenylpropinyl)porphyrin}
\subsection{1.Stufe: 5-(Pentafluorophenyl)dipyrromethan}
In einen 50 mL 2-Halskolben werden 25 mL (361,45 mmol) Pyrrol vorgelegt, 2,69 g (13,55 mmol) Pentafluorobenzaldehyd beigemengt und 15 min unter Argonfluss entgast. Mittels Pipette wird 105 \textmu L (1,36 mmol) Trifluoressigsäure zugefügt und 5 min bei Raumtemperatur unter Argonfluss gerührt. Dann wird mit 1 mL 1 M NaOH-Lösung gequencht (pH-Wert überprüfen!) und die Lösung mit 100 mL Ethylacetat in einen Schütteltrichter überführt. 4 mal wird mit jeweils 50 mL Wasser (ein paar Tropfen NaCL-Lösung werden zugefügt) extrahiert. Die organische Phase wird über Na$_2$SO$_4$ getrocknet und mit Filterpapier abfiltriert. Überprüfung mittels DC. Das Lösungsmittel wird am Rotavapor abrotiert und bei Vollvakuum getrocknet. Aufnahme eines NMR-Spektrums zur Reinheitsüberprüfung. Die Aufreinigung erfolgt mittels Kieselgelchromatographiesäule. Die Reinheitskontrolle erfolgt mittels DC. Die Fraktionen mit Produkt werden einrotiert und ein NMR angefertigt. Mittels Ausschütteln mit 40\% iger Ethanol-Lösung (3x 20mL) wird das restliche Pyrrol entfernt. Der Kolben wird im Vakuumtrockenschrank über Nacht getrocknet.
%H-NMR


\subsection{2.Stufe: bis(PF$_5$)bis(PhenylPropinyl)porphyrin}

In einen gekühlten trockenen 500 mL 2-Halskolben mit Argonzufluss werden 515 mg 5-(Pentaluorophenyl)dipyrromethan (1,65 mmol) eingewogen und in 250 mL $CH_2Cl_2$ gelöst. Dann werden 202 \textmu L Phenylpropinyl (1,65 mmol, 215 mg) hinzugefügt. Die Lösung wird 10 min im gekühlten Ölbad gerührt. Das Ölbad wird danach entfernt und das Gefäß mit Alufolie umwickelt und 120 \textmu L $BF_3C_4H_8O$ (0,99 mmol) hinzugefügt und 4 h rühren lassen. Danach werden 800 mg DDQ (3,30 mmol, 2 Äquivalente) hinzugefügt und über Nacht gerührt.
\\ Am nächsten Tag gibt man 1 mL Triethylamin hinzu und rührt. Dann wird das Lösemittel abrotiert und der Kolben im Vollvakuum getrocknet. Reaktionskontrolle mittels DC und Spektrometer ergibt, dass sich kein Produkt gebildet hat, da keine typische Absorption des Porphyrins im Spektrum erkennbar ist.

\section{tetra(propinyl)tetra(3,4 diethylpyrrol)}

In einen 500 mL Zweihalskolben mit Stickstoffzufluss und Kühlbad werden in ca. 400 mL Dichlormethan bei -40°C (Aceton/Stickstoff) 490 \textmu L Phenylpropinyl und 492 mg 3,4-diethylpyroll (4mmol, 1 Äqu.)gegeben und die Apparatur mit Alufolie umwickelt. Nach ca. 0,5 h werden 160 \textmu L BF$_3*OEt_2$ hinzugegeben. Nach 4 h lässt man auf Raumtemperatur abkühlen. Zur Reaktionskontrolle wird ein UV-vis-Spektrum aufgenommen. Da nach Zugabe von 908 mg DDQ (4 mmol) kein typsches Absorptionsspektrum eines Porphyrins erkennbar ist, wird die Reaktion abgebrochen.


\section{tetra(methylphenyl)TBP}
\subsection{1.Stufe: Zinktolylacetat}

2,50 g Tolylessigsäure (16,7 mmol, 2 Äqu.) werden mit 680 mg Zinkoxid (8,35 mmol, 1 Äqu.) vermischt und mit 40 ml EtOH und 20 mL H$_2$O bei 80°C 14 min gerührt. Das Lösungsmittel wird abrotiert und das Produkt bei 70°C und Vollvakuum getrocknet.

\subsection{2.Stufe: Zn-tetra(methylphenyl)TBP}

In einem Becherglas werden 3,84 g 1,2-dicyanobenzene (30 mmol, 4 Äqu.), 4,50 g Tolylessigsäure (30 mmol, 4 Äqu.) und 2,75 g Zinktolylacetat (7,5 mmol, 1 Äqu.) eingewogen und in einem Mörser mittels Pistill vermischt. Die Mischung wird in 15 Supelco Vials aufgeteilt. Die Vials werden im Heizblock auf 280$\circ$C aufgeheizt und für 10 min erhitzt. Die dunkelgrüne Lösung wird abgekühlt. Die Vials werden mit Aceton aufgefüllt und 10 min im Ultraschallbad erwärmt. Dann werden  die Lösungen in einem Becherglas zusammengeschüttet. Eine 1000 mL Lösung bestehend aus 3 Teilen EtOH und 2 Teilen und jeweils 25 mL NaHCO$_3$ und NaCl wird vorbereitet und die Porphyrin-Lösung langsam hinzugegeben (wird gefällt). Der grüne Feststoff wird abfiltriert. Der Vorgang (in Aceton lösen, Ausfällung) wird wiederholt und der Feststoff über Nacht getrocknet.
\\Der Farbstoff wird in Aceton gelöst und eine Säulenchromatographie (Säulenmaterial: Al$_2O_3$) durchgeführt. Der Farbstoff wird mit Cyclohexan und Dichlormethan eluiert. Die Fraktionen, die den Farbstoff enthalten, werden am Rotavapor einrotiert, 2 mal mit Hexan nachgewaschen, 1x mit  Dichlormethan extrahiert und der grüne Farbstoff im Trockenschrank getrocknet.

\subsection{3.Stufe: Demetallierung zu tetra(tolyl)tetrabenzoporphyrin}
0,400 g Zn-tetra(methylphenyl)tetrabenzoporphyrin werden in 200 mL DCM gelöst und 10 min gerührt. Die Lösung wird in einen Schütteltrichter transferiert. Dann wird mit konzentrierter Salzsäure (2x 100 mL), 3x mit jeweils 100 mL destillierten Wasser und zweimal mit 150 mL gesättigter NaHCO$_3$-Lösung extrahiert. Die organische Phase wird über Na$_2$SO$_4$ getrocknet, abfiltriert und das Lösungsmittel mittels Rotavapor entfernt. Das Produkt wird über Nacht bei 60°C getrocknet. Da das Produkt  schlecht lösbar ist, (als Farbstoff ungeeignet), wird hier die Reaktion abgebrochen.

\section{Pt(II)-mono(O-Ethylphenyl)-tris(fluorophenyl)tetrabenzoporphyrin}
\subsection{1.Stufe: Zn-mono(O-Ethylphenyl)-tris(fluorophenyl)tetrabenzoporphyrin}
5,13 g Dicyanobenzene (40 mmol, 4 Äqu.), 2,70 g 4-Ethoxyphenylessigsäure (15 mmol, 1,5 Äqu.),3,85 g 4-Fluorphenylessigsäure (25 mmol, 2,5 Äqu.) und 3,71 g ZnC$_16$H$_12$F$_2$O$_4$ (10 mmol, 1 Äqu.) werden miteinander vermischt und in 22 Supelco-Vials aufgeteilt. Diese werden bei 280$\circ$C für 10 min erhitzt. Nach Abkühlen des Heizblocks werden die Vials mit Aceton aufgefüllt und in ein Becherglas überführt (unlösliche Bestandteile werden mittels Ultraschallbad gelöst). Der Farbstoff wird in 1200 mL (3 Teile EtOH, 2 Teile Wasser, gesättigte NaHCO$_3$-Lösung) ausgefällt und dann abfiltriert. Das Filtrat wird über Nacht in den Trockenschrank gestellt. Die Aufreinigung erfolgt mittels Säulenchromatographie (Säulenmaterial: Al$_2$O$_3$). Die geeigneten Fraktionen werden einrotiert. Der Feststoff wird in Dichlormethan gelöst und in Cyclohexan ausgefällt, abfiltriert und im Vakuumtrockenschrank getrocknet.


\subsection{2.Stufe: Demetallierung zu monoOEttriFPhenTBP}
Das Produkt wird in Dichlormethan gelöst und eine Demetallierung durchgeführt. Die Lösung wird 2 mal mit 36\% Salzsäure und dreimal mit Wasser extrahiert. Nach Extraktion mit gesättigter NaHCO$_3$-Lösung und Wasser wird die organische Phase über Na$_2$SO$_4$ getrocknet und abfiltriert. Das Lösungsmittel wird am Rotavapor abgezogen.

\subsection{3.Stufe: Platinierung}
0,5 mmol (ca. 450 mg, 1 Äqu.) des demetallierten Komplexes werden in einem Rundkolben in 200 mL Trimethylbenzen gelöst und mit 250 mg cis-Platin (1 mmol, 2 Äqu.) vermengt und auf 170°C erhitzt. Reaktionskontrolle erfolgt über Aufnahme von Spektren. Nach 1 Stunde wird 150 mg Platinkomplex hinzugegeben. Nach 1 weiteren Stunde wird die Reaktion abgebrochen und auf Raumtemperatur abgekühlt. Dann wird das Lösungsmittel abgezogen. Die Reinigung erfolgt mittels Säulenchromatographie( Säulematerial: Al$_2$O$_3$, eluiert wird mit Cyclohexan und Dichlormethan). Die Trennung des Produktes von seinen Isomeren (di- und nicht-substituierte Nebenprodukte) konnte nicht erreicht werden.

\section{Verbrückung von Pt(II)tertbutylbenzoporphyrin mittels Schollreaktion}
In einen Rundkolben werden 15 mg Tertbutylbenzoporphyrin und 120 mg AlCl$_3$ in 15 mL Dichlorbenzen gelöst. Der Kolben wird 5 min im Ultraschallbad erhitzt. Dann wird im Ölbad auf 170 $\circ$C erwärmt und 15 min gekocht. Danach werden 50 mg MoCl$_5$ zugegeben und 3 min weiter erhitzt. Danach wird die Reaktion abgekühlt. Die Lösung wird in 150 mL DCM und 4 mL Et$_3$N gegeben und zentrifugiert. Das Sediment wird 2 mal nachgewaschen. Dann wird die Lösung dreimal mit NaCl-Lösung extrahiert. Die wässrige Lösung wird einmal mit DCM nachgewaschen.  Die vereinigten organischen Phasen werden über Na$_2$SO$_4$ getrocknet. Die Reaktionskontrolle erfolgt über ein DC( Laufmittel: CH:DCM=1:1). Zur Aufreinigung wird eine Säulenchromatographie gemacht. Das fertige Produkt wird in Toluol gelöst und ein Fluoreszenzpektrum aufgenommen. Die Reaktionskontrolle erfolgte über das UV-vis-Spektrum. Durch Vderscheibung der Peaks nach rechts konnte das Produkt nachgewiesen werden.

\section{Verbrückung von PttBumonoAzaTBP mittels Scholl Reaktion}
2mg PtAzaporphyrin werden mit 15 mg $AlCl_3$ und 5 mg $MoCl_5$ in 2 mL Dichlorbenzen gelöst und im Heizblock in einem Vial bei 170$\circ$C 15 min erhitzt.Nach Abkühlen der Lösung wird diese in 30 mL Dichlormethan und 5 mL Triethylamin gegeben und abfiltriert. Dann wird 3x mit 50mL $H_2O$ extrahiert, über $Na_2SO_4$ getrocknet, abfiltriert. Danach wird das Dichlormethan abgedampft und ein Fluoreszenzspektrum aufgenommen:
