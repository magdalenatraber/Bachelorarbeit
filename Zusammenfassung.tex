Die Synthese der Porphyrine erwies sich als schwierig. Beschriebene Synthesen funktionierten nicht, weil keine stabilen Produkte gebildet werden konnten. Die Löslichkeit der Porphyrine war sehr gering, da sie große planare Strukturen bilden. Manchmal konnte man auch keine Reinigung mittels Säulenchromatographie gewährleisten.
Die gewonnene Erkenntnisse helfen aber in Zukunft, neue Strategien zu entwickeln und die besonderen Eigenschaften der Porphyrine zu berücksichtigen.
\\Sehr vielversprechend war die Verbrückung von Benzoporphyrinen mittels Schollreaktionen. Diese ergab Produkte, die im nahen Infrarotbereich emittieren und damit als Indikatoren für optische Sensoren geeignte sind. Der Einsatz eines Gemisches aus $Al_2O_3$ und MoCl$_5$ erwies sich als effektivsten. Der genaue Mechanismus dieser Reaktion ist noch nicht geklärt. Auch die genaue Struktur der verschiedene verbrückten Porphyrine muss noch erforscht werden, was im Umfang dieser Bachelorarbeit leider nicht möglich war. 